\subsection{VertexAssociators}


\verb/VertexAssociator/ classes are classes which associate simulated and
 reconstructed vertices, which is essential to evaluate vertex
 reconstruction efficiencies and the fake rates of the various
 reconstructors.
Their abstract base class is \verb/VertexAssociator/, and the relevant methods are:

\begin{verbatim}
  vector<const RecVertex *> recVertices(const TkSimVertex&)
  vector<const TkSimVertex *> simVertices(const RecVertex&)
  \end{verbatim}

These return the appropriate container of vertices which are associated to
 the input vertex.
If more than one vertex is associated, the first element of the container
 ({\tt front()}) is the best association.

The two concrete implementations provided in the package 
\verb/VertexAssociation/ are:

\begin{itemize}
\item {\tt VertexAssociatorByTracks}: \\
 associates two vertices if the fraction of tracks in common is greater or
 equal a threshold set by\\ {\tt
 VertexAssociatorByTracks:FractionOfAssociatedTracks}.\\ 
 The default value is 50\%, meaning that for vertices with two good
 tracks, one associated track is enough.
 Only reconstructed {\tt TkSimTracks}, i.e.\ associated to a
  \verb/RecTracks/,  are considered in the fraction,
  thus defining an algorithmic efficiency. 
  The track associator to use, and the collection of all simulated and
 reconstructed vertices has to be given at construction:
  \begin{verbatim}
VertexAssociatorByTracks(const vector<const TkSimVertex *> &, 
  const vector<const RecVertex *> & ,const TrackAssociator &)
  \end{verbatim}

\item {\tt VertexAssociatorByDistance}: \\
associates vertices according to the distance between them.
The associated vertices are ordered by increasing order of distance. 
The algorithm to compute that distance (as an implementation of
 \verb/VertexDistance/~\ref{vertex_distance}) has to be given with the method:
\begin{verbatim}
set_distance(VertexDistance * vertexDistance)
\end{verbatim}

\end{itemize}

Usage example for the  {\tt VertexAssociatorByTracks}:
\begin{verbatim}
  vector<const TkSimVertex *> theVectorOfSimVertices = ...
  vector<const RecVertex *> theVectorOfRecVertices = ...

  TdrTrackAssociatorByHits trackAssociator(simTracks, recTracks);

  VertexAssociatorByTracks vertexAssociator(theVectorOfSimVertices,
    	      theVectorOfRecVertices, trackAssociator);

  vector<const TkSimVertex *> assoSimV =
  vertexAssociator.simVertices(aRecVertex);

\end{verbatim}

The class \verb/VertexAssociationToolsFactory/ creates consistent
 VertexAssociator and TrackAssociator objects for estimation of vertex
 finding performance.
It allows to choose between Associators at runtime via
 the SimpleConfigurable mechanism.
The SimpleConfigurables used are called:

\begin{itemize}
\item \verb/VertexAssociationToolsFactory:VertexAssociator/:\\
Selects the type of \verb/VertexAssociator/, and 
 takes either \verb/ByTracks/ or \verb/ByDistance/ as its argument.
\item \verb/VertexAssociationToolsFactory:TrackAssociator/:\\
Selects the type of \verb/TrackAssociator/, and 
 takes either \verb/ByHits/ or \verb/ByPulls/ as its argument.
\end{itemize}

% Is this for a single event only?

\subsection{Typedefs}
Some typedefs of general use are defined within the name-space
\verb/VertexAssociator/:
%
\begin{verbatim}
typedef vector$<$const RecVertex*$>$ RecVertexPtrContainer;
typedef vector$<$const TkSimVertex*$>$ SimVertexPtrContainer;
typedef vector$<$const RecVertex$>$ RecVertexContainer; 
typedef vector$<$const TkSimVertex$>$SimVertexContainer;
\end{verbatim}
%
and should always be referenced as in
\verb/VertexAssociator::SimVertexContainer/.
